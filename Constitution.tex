\documentclass{article}
\usepackage{enumitem}
\title{Pork Rub Fantasy Football Constitution}
\date{Ratified May 30, 2020}
\begin{document}
\begin{titlepage}
    \maketitle
    %Hello world!
\end{titlepage}
\begin{center}
    \section*{PREAMBLE}
\end{center}

\textit{We, the owners of the Pork Rub Fantasy Football League, in order to form a more perfect league, establish fairness, insure inter-owner tranquility, provide for the right to veto by majority, promote the general welfare of all teams, continue our commitment to excellence, and secure the legitimacy of the league, do ordain and establish this constitution of the Pork Rub Fantasy Football League.}

\newpage
\begin{center}
    Copyright \copyright{} Pork Rub Fantasy Football 2017. Last edited May 30, 2020
\end{center}

\newpage
\tableofcontents
\newpage
\section{Article I. Executive Positions}
    \noindent\rule{\textwidth}{0.5pt}\\

    \noindent\textit{This section broadly describes the official executive positions of The League with each of their respective power and duties. All executive positions except the League Manager and Ombudsman will be determined at the annual Constitutional Convention as described in \textbf{Article V}.}
    \subsection*{\textit{Section 1. Office of the League Manager}}
    The League manager, or “Commissioner”, is the head of state of The League. This position, appointed by the current Commissioner, holds the ultimate authority over The League. This position comes with the following duties to be upheld throughout a season and offseason:\\

    \noindent\textbf{General Powers and Duties}
    \begin{enumerate}[label=\Alph*)]
        \item Uphold, protect, and be faithful to the Constitution.
        \item Oversee and execute the Constitutional Amendment process.
        \item Sign and ratify or let pass into law Amendments to this Constitution.
        \item Oversee and execute the Constitutional Convention.
        \item Enforce the rulings of the Justices of the Supreme Pork Court.
        \item Retains discretion to enforce the terms of punishment as issued by the Supreme Pork Court to their fullest or lesser extent, but not beyond their fullest extent.
        \item Execute and administer the Game of the Week each week.
        \item Update Power Rankings each week to reflect the current state of The League.
        \item Organize the Draft order and Draft day to allow the maximal number of owners to be present and minimize the amount of autodrafting.
        \item When deemed necessary, allocate League Manager powers to the Vice-Commissioner to execute powers of the League Manager’s office, as dictated by \textbf{Article I.2}.
        \item Manually change an owner’s team if, and only if, given the authority to by the team owner. The Commissioner is required to address The League prior to any manipulation.
        \item Unenumerated powers reserved for the Commissioner.
    \end{enumerate}
    \subsection*{\textit{Section 2. Office of the Vice-Commissioner}}
    The position of Vice-Commissioner is second in command after the League Manager. There is no limit on consecutive terms, and therefore a single owner may be elected to multiple years in office. In the instance of the League Manager being unable to fulfill his duty for whatever reason, the League Manager can temporarily allocate the necessary authority to the Vice-Commissioner to execute certain powers of the League Manager office. The League Manager must notify The League prior to the temporary transfer of the powers as described below. These include and are limited to:
    \begin{enumerate}[label=\Alph*)]
        \item Writing and uploading the Game of the Week, including any predictions related to the GOTW or The League in general.
        \item Writing post-week updates, or other necessary updates regarding The League to the main page.
        \item Manually change an owner’s team if, and only if, given the authority to by the team owner.
        \item Inheritance of other League Manager powers and duties as explicitly allowed by the Commissioner’s wishes.
    \end{enumerate}
    \subsection*{\textit{Section 3. Office of the Ombudsman}}
    The Office of the Ombudsman has been created to provide independent and impartial investigative oversight into matters thought to involve collusion, bribery, bad-faith dealings, or other actions either prohibited by the Constitution or contrary to the spirit of Pork Rub. The Ombudsman is appointed by the Justices of the Pork to investigate these matters as they arise. When summoned, the Ombudsman has the authority to conduct fact finding missions, conduct interviews, collect evidence, or perform other investigative actions in the service of his mission. In an advisory capacity, the Ombudsman may offer his opinion concerning guilt or innocence and he may recommend punishment. However, the Ombudsman has no authority to declare guilt nor innocence, nor does he retain the authority to level punishment.
    \begin{enumerate}[label=\Alph*)]
        \item Office is filled on a case-by-case basis by the Justices of the Pork.
        \item The Commissioner will announce the appointment of the Ombudsman as decided by the Justices within 24 hours.
        \item Appointee shall be a neutral party in relation to the matter being investigated.
        \item The Ombudsman must surrender his office upon the conclusion of his investigation.
        \item The Commissioner nor current Justices of the Pork are eligible to be appointed.
    \end{enumerate}
    \subsection*{\textit{Section 4. Office of the Comptroller}}
    The Comptroller controls the monetary flow in The League. In years past, many owners have not paid on time, often well past the end of the previous year. This is not only unfair to the winners, but the Commissioner, who has often had to pay out of pocket. Thus, the Office of the Comptroller, whose sole purpose is to collect fees and distribute payouts, is now established. Fees due are dictated by \textbf{Article III.8}. The duties include:
    \begin{enumerate}[label=\Alph*)]
        \item Enforce the payment policy and ensure that every member pays dues.
        \item Be in contact with the Commissioner to ensure everyone has paid and been paid when necessary.
        \item It is the Comptroller’s solemn duty to obtain payment and deliver payout as necessary and with haste.
    \end{enumerate}

\section{Article II. Justice of the Pork}
    \noindent\rule{\textwidth}{0.5pt}\\

    \noindent\textit{As a league that seeks justice and tranquility for all, it is necessary to designate a judiciary branch of The League. The article establishes three positions – three Justices that sit on the bench of the Pork Rub Fantasy Supreme Pork Court.}
    \subsection*{\textit{Section 1. Selection, Criteria, and Positions of Justices}}
    It is in the best interest of The League to have a diverse and neutral judicial committee in the event of conflicts of interest of other positions of power. With this, the following are the law of the judicial court:
    \begin{enumerate}[label=\Alph*)]
        \item The court will consist of three owners that will hold the positions of \textit{Justice of the Court}.
        \item A Justice may not hold any other office simultaneously.
        \item The tenure of a Justice is three years; consecutive (partial) terms are not allowed.
        \item The Justices’ terms shall be staggered.
        \item Justice nominees to fill openings on the Supreme Court will be selected at the Constitutional Convention.
        \item The Court will consist of representatives from all three (3) divisions.
    \end{enumerate}
    \subsection*{\textit{Section 2. Duties of Justices}}
    The duties of a Justice include the following:
    \begin{enumerate}[label=\Alph*)]
        \item Rule on questions of Constitutional interpretation and due process.
        \item Grant fair and swift hearings to owners with grievances.
        \item Issue terms of punishment to parties at fault. However, the Court has no power to enforce the punishment. Enforcement of the terms of punishment lie with the Commissioner’s office.
        \item Investigate the Commissioner and other Executives if necessary (\textbf{Article VI.2}).
    \end{enumerate}
    \subsection*{\textit{Section 3. Retention Elections}}
    Justices are vested with great power and their terms exceed every other except the Commissioner’s life term. Therefore, it is prudent that a mechanism be provided to allow the premature removal of a Justice if needed as judged by the owners of Pork Rub.
    \begin{enumerate}[label=\Alph*)]
        \item During the May voting round, a Proposition may be submitted to The League for vote on the retention of a Justice. The process for bringing this to a vote before The League must follow the Constitutional Amendment process, with the exception of \textbf{Article IV.1.B}. This, however, does not constitute a Constitutional Amendment.
        \item A supermajority must vote in favor of the Justice’s removal to remove him from office.
        \item If the removal vote is sustained, then the vacant seat will be filled at the subsequent Constitutional Convention.
    \end{enumerate}
\section{Article III. Rules of the Game}
    \subsection*{\textit{Section 1. Rosters}}
    \subsection*{\textit{Section 2. Draft}}
    \subsection*{\textit{Section 3. Divisions}}
    \subsection*{\textit{Section 4. Schedule}}
    \subsection*{\textit{Section 5. Waiver Wire}}
    \subsection*{\textit{Section 6. Point Scoring}}
    \subsection*{\textit{Section 7. Playoffs}}
    \subsection*{\textit{Section 8. Monetary Dues and Payout}}
    \subsection*{\textit{Section 9. Last Place Punishment}}
    \subsection*{\textit{Section 10. Trading of Players and Draft Picks}}
    \subsection*{\textit{Section 11. Trade Vetoing}}
    \subsection*{\textit{Section 12. The Commitment to Excellence Award}}
    \subsection*{\textit{Section 13. Trade Time}}
    \subsection*{\textit{Section 14. Side-pot Rules}}

\section{Article IV. Mode of Amendment}
    \subsection*{\textit{Section 1. Year-round Rules}}
    \subsection*{\textit{Section 2. Offseason Amendments}}
    \subsection*{\textit{Section 3. In-season Amendments}}

\section{Article V. Constitutional Convention}
    \subsection*{\textit{Section 1. Filling Official League Positions}}

\section{Article VI. Special Cases}
    \subsection*{\textit{Section 1. Death Penalty}}
    \subsection*{\textit{Section 2. Investigation of Executives}}
    \subsection*{\textit{Section 3. Prospective Pork Rub Owners}}
    \subsection*{\textit{Section 4. The Great Powers}}

\end{document}